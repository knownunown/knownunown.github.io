\documentclass{moderncv}
\usepackage{fontawesome5}
\usepackage{xcolor}
\usepackage{graphicx}
\usepackage{enumitem}
\usepackage{tikz}

\newcommand{\ghsubsection}[1]{\subsection{\href{https://github.com/tnytown/#1}{#1\texorpdfstring{ \faGithub}{}}}}

\newcommand{\comment}[1]{}

\moderncvtheme[blue]{banking}
\nopagenumbers{}

%% LaTeX font stuff: serif font
\usepackage{ebgaramond}
\renewcommand{\sfdefault}{\rmdefault}

\usepackage[utf8]{inputenc}
\usepackage[scale=0.9]{geometry}
\usepackage{luacode}

%% Personal data
\firstname{Andrew}
\familyname{Pan}

\title{computation enthusiast}
\homepage{tny.town}
\email{hire@tny.town}
\social[github]{tnytown}
\social[linkedin]{tny}
\extrainfo{Modified \lastmodified, Git revision\githash \makeheaddetailssymbol \href{https://tny.town/cv.pdf}{tny.town/cv.pdf}}

\begin{document}
\begin{luacode}
require("lualibs.lua")

local f = io.open('rev.json', 'r')
local s = f:read('*a')
f:close()

print(s)
nfo = utilities.json.tolua(s)
rev = nfo.rev
dat = os.date('\%b \%Y', nfo.modified)
\end{luacode}

\newcommand{\githash}
{
  \href{https://github.com/tnytown/website/blob/\luadirect{tex.sprint(rev)}/cv.tex}
  {\texttt{\luadirect{tex.sprint(rev)}}}}
\newcommand{\lastmodified}{\luadirect{tex.sprint(dat)}}

\makecvtitle{}
\vspace*{-15mm} % decrease header padding

\section{Education}
\cventry{2019--present}
{Bachelor of Arts in Computer Science}
{University of Minnesota}
{}{}
{
  Currently \textbf{4th year}. Dean's List S2020-F2021. Waller Scholarship recipient 2021. Pursuing French Studies and Philosophy minors.
}

\section{Experiences}
\cvline{}{\textit{Further details and work samples available \href{mailto:hire@tny.town}{\underline{on request}}.}}

\cventry
    {May--August 2022}{Software Engineer Intern}{CrowdStrike}{}{}
    {
      \begin{itemize}[label=\rightarrow,noitemsep]
      \item Instrumented Go service and built Grafana dashboard to expose per-endpoint response metrics.
      \item Retooled Makefile-based build system to support integration of Rust library into Go service.
      \item Developed, tested, and deployed new features in Go services with Jenkins, Kubernetes, and Spinnaker.
      \end{itemize}
    }

\cventry{2022--present}{Teaching Assistant}{College of Science and Engineering}{}{}
{
  \begin{itemize}[label=\rightarrow,noitemsep]
  \item Provided feedback on student assignment submissions, conducted weekly labs.
  \end{itemize}
}
\cventry{2021--present}{Undergraduate Research Assistant}{Adamala-Engelhart Lab}{}{}
{
  \begin{itemize}[label=\rightarrow,noitemsep]
  \item Maintaining a predominantly Python Flask+SQLAlchemy app to enable research with synthetic biology of cyanobacteria.
  \item Retrofitted a novel and intuitive VueJS interface on top of the Flask app.
  \item Exploring novel applications of computing to synthetic biology research.
  \end{itemize}
}
   
\cventry{2020--present}{Generalist, Webmaster, Vice President}{ACM UMN}{}{}
{
  \begin{itemize}[label=\rightarrow,noitemsep]
  \item Planned the first and only online MinneHack (Minnesota's largest hackathon) with the ACM officers. Prepared promotional materials and executed a non-traditional hackathon structure.
  \item Maintained web presence for the ACM UMN, static sites checked with GitHub Actions CI and Nix.
  \item Conducting events for students interested in computing, including UNIX classes and Capture The Flag.
  \end{itemize}
}

\comment{
\cventry{2019--2020}{Member, Controls Team}{UMN Solar Vehicle Project}{}{}
{
  \begin{itemize}[label=\rightarrow,noitemsep]
  \item Worked on design of the rearview system and integrated USB tethered logging for the telemetry system.
  \item Prototyped an editor for a YAML-based CAN packet definition schema in Rust.
  \item Participated extensively in build cycle (layups) for the Freya vehicle, which won 1st place at the Formula Sun Grand Prix.
  \end{itemize}
}}

\section{Trivia}
\cvline{}{
  \textit{Ask me about the following things.}
  \pdfliteral page{q 3 Tr}{This section is mostly for keyword scrapers.}\pdfliteral page{Q} % https://tex.stackexchange.com/questions/463968/hide-text-from-displaying-but-retain-it-selectable-and-searchable
}

\cvdoubleitem
{web}
{
  REST, PHP, Node, HTML/JS
}
{sysadmin}
{
  NixOS, UNIX-like OSes, awk, sed, Docker
}

\cvdoubleitem
{mobile}
{
  Java, Android apps, Frida
}
{misc}
{
  Git, CI/CD, Reverse Engineering, r2, Go, C/C++
}

\section{Personal Projects}
\cvline{}{
  \textit{\href{https://github.com/knownunown}{\faGithub\ \underline{Check out more on my GitHub profile}!}}
}

\ghsubsection{website}
\cvitem{keywords}{blog, writing, nix}

\cvitem{}{Personal blog with posts related to software.}

\comment{
\ghsubsection{imsUS}
\cvitem{keywords}{reverse engineering, rust, telephony, networking}
\cvitem{}{Reverse engineered ``Call from iPhone'' stack on macOS. Reimplemented
  IKEv2/EAP-TLS, IMS, and SIP registration to send and recieve text messages
  over WiFi with mobile number from commodity computers.}
}

\ghsubsection{tpl-c900-openwrt-install}
\cvitem{keywords}{reverse engineering, embedded, unix, lua, go}
\cvitem{}{Broke firmware verification scheme on the TP-Link Archer C900, a consumer
  router. Installed open source router firmware (OpenWRT). Documented process.}

\ghsubsection{u2dl}
\cvitem{keywords}{web, fish, unix, ffmpeg}
\cvitem{}{Developed Fish shell script to download YouTube videos. Bypassed the ``rolling cipher''
  with standard Unix tools. Wrote an accompanying minifier to compress the script down
  \char`\~291\%, final size 2kb.}

\comment{
\ghsubsection{J2V8}
\cvitem{keywords}{cpp, nodejs, android, docker}
\cvitem{}{Ported J2V8, Java bindings for the NodeJS JavaScript engine, to
  Android. Automated build process with a Docker image. Submitted changes to upstream.}
}
\end{document}
